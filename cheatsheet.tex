\documentclass{article}

\usepackage[utf8]{inputenc}
\usepackage{amsmath}
\usepackage{amssymb}
\usepackage{ bbold }

\begin{document}

\section{Operatoren}
\subsection{Eigenschaften}
Ein Operator $A$ ist
\begin{description}
\item[Unitär]:
  \[ A\cdot A^\dagger = 1 \]
  Es gilt:
  \[ \| A\cdot x \| = \| x \| \]
\item[Hermitesch]:
  \[ A = A^\dagger \]
\end{description}
\subsection{Hadamard}

\[ H = \frac{1}{\sqrt{2}}\left(\begin{array}{rr}
    1 &  1\\
    1 & -1
  \end{array}\right) \]
\subsection{Pauli-Matrizen}
\[ \sigma_x = \left(\begin{array}{rr}
    0 & 1\\
    1 & 0
  \end{array} \right) \]
\[ \sigma_y = \left(\begin{array}{rr}
    0 & -i\\
    i & 0
  \end{array} \right) \]
\[ \sigma_z = \left(\begin{array}{rr}
    1 & 0\\
    0 & -1
  \end{array}\right) \]
Es gilt:
\[ \sigma_k^2 = 1 \]
\[ \sigma_j\sigma_k = -\sigma_k\sigma_j \]
\[ \sigma_x\sigma_y = i\sigma_z \]
\[ \sigma_y\sigma_z = i\sigma_x \]
\[ \sigma_z\sigma_x = i\sigma_y \]

Pauli-Matrizen sind unitär d.h. 
\[ A \cdot A^{\dagger} = \mathbb{1} \]


Allgemeine Heisenbergsche Unschärferelation:
\[ \Delta A \cdot \Delta B \ge \frac{1}{2} |\left<\psi|[A,B]|\psi\right>| \]

  
\subsection{Cnot}
\[ Cnot = \left(\begin{array}{rrrr}
    1 & 0 & 0 & 0\\
    0 & 1 & 0 & 0\\
    0 & 0 & 0 & 1\\
    0 & 0 & 1 & 0
  \end{array}\right) \]

\subsection{Skalarprodukt}
\[ \left<u|v\right> = \overline{\left<v|u\right>} \]
\[ \left<\lambda\cdot u|v\right> = \overline{\lambda}\left<u|v\right> \]
\[ \left<u|\lambda\cdot v\right> = \lambda\left<u|v\right> \]
\[ \left<x|y+z\right> = \left<x|y\right> + \left<x|z\right> \]
\[ \left<x+y|z\right> = \left<x|z\right> + \left<y|z\right> \]
\end{document}
